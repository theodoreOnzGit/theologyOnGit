\documentclass[11]{article}

\usepackage[autostyle]{csquotes}

\usepackage[
    backend=biber,
    style=authoryear-icomp,
    sortlocale=de_DE,
    natbib=true,
    url=false, 
    doi=true,
    eprint=false
]{biblatex}
\addbibresource{historicityOfScripture.bib}


% to use hebrew
\usepackage{cjhebrew}

\begin{document}


\part{The Question on Hand}

\section{Introduction}

Genesis often comes up as a matter of debate among christans, scholars and critics. The six day creation account has generated plenty of controversy as to the method of interpretation of Genesis 1-3 and therefore how much of it can be trusted. 

It is the prevailing scientific thought that the universe is billions of years old and the earth is several billion years old. Earth was formed through gradual processes, and life somehow evolved as a matter of a combination of abiogenesis and evolution. Evolution attempts to explain the many forms and types of life present on planet earth while abiogenesis attempts to explain how the first cell was formed.


Such prevailing thoughts seem to contradict the account given in Genesis. Thus generating the said controversy.

\section{Scope}

Seeing how the subject matter is so broad, it is important to narrow the search down to several specific questions:

\begin{itemize}
\item What are the prevailing thoughts of genesis interpretation?
\item What is the prevailing scientific consensus of the origin of the earth and life? What methods were used to arrive at such a conclusion?
\item What are the implications of the various forms of genesis interpretation?
\item What method of interpretation is most plausible?
\end{itemize}

\section{disclaimer}

I am human and there will be biases present. I am saying these upfront. As of the start of this research, I am more inclined to believe in a miraculous literal 24 hr creation account having experienced physical miracles personally. I am also keen to look into scientific evidence. The age of the earth is one of them, where it seems K-Ar dating dates the earth to several billion years old. Thus i am more inclined initially to have a variation of gap theory. Earth would be created several billion years ago, formless and empty with water. Thus, elements with their radioisotopes would be present. And God simply terraformed the earth within six 24hr days.

However, i acknowledge this has weaknesses. Why would God wait several billion years with a ball of rock and water? 

This most closely resembles gap theory. However, i do not believe earth was created good and then became formless and void, but was created formless and void. As to why, i don't really know. 

\part{Prevailing thoughts of Genesis Interpretation}

We start by examining previous scholarly thought much like a literature review. Not all of these come from peer reviewed articles unfortunately, but they do represent the various worldviews held by many christians.

\section{biologos review}

\begin{verbatim}
https://biologos.org/articles/comparing-interpretations-of-genesis-1/
\end{verbatim}

According to this website, there are so far eight non exhaustive methods of interpreting genesis 1.

\subsection{Young Earth Interpretation}

Creation occured 6000 years ago, during six 24h days literally as described. 

\subsection{Gap Interpretation}

In Genesis 1:1-2, we see that the earth was formless and void. The earth was originally okay. Then "became" formless and void, and was restored over six literal 24hr days. One might view the age of the rocks being very old, (cite here). 

\subsection{Day-Age Interpretation}

The word day or yom \<ywm> should not be interpreted as 24 hr periods, but rather as long epochs lasting billions of years long.

\subsection{Appearance of Age Interpretation}

Earth was created as is in six 24 hr days along with the entire universe. The universe was functional and mature as is. However, with prevailing scientific theories and dating methods, we might view this to be billions of years old.

\subsection{Proclamation Day Interpretation}

Time was relativistic between God's throne room and earth. The day here refers to days in God's frame of reference and not from earth's frame of reference.

\subsection{Creation Poem Interpretation}

Genesis 1 is to be interpreted like a poem, and not literally.

\subsection{Kingdom and Temple Interpretation}

Similar to Proclamation Day interpretation, the days and events here do not refer to earth directly, but from God's perspective. The text does not focus on this physical universe. God then gives humans a "land grant" covenant (i have little idea what this is, though i take it to mean that God just leases land to human beings, calling them to be fruitful and multiply.)

\subsection{Ancient Near Eastern Cosmology Interpretation}

Genesis 1 was written to match the context of the ancient world, where they had a certain concept of how the earth was created. Genesis 1 was written for the ancient audience and not modern scientific audience.

Using this as a starting point, we now compare it to other sources which say how christians might interpret genesis. 

\section{Gospel Coalition review}

\begin{verbatim}
https://www.thegospelcoalition.org/essay/evangelical-interpretations-genesis-1-2/
\end{verbatim}

Firstly, the young earth creationism view is exactly the same as above. Six literal 24 h days with no gap between earth being formless and void and then being created on by God.

Secondly mature creation theory closely correlates to the appearance of age interpretation. For which God creates this universe as is, seemingly mature. 

Thirdly, revelatory day theory is discussed. The six 24 hr days refer to six 24 hr days where God revealed creation to Moses, not six 24hr days of creation on earth.

Fourthly, gap theory correlates to gap theory above. And the cause of the earth being formless and void was due to the fall of Satan. 

Fifth, local creation theory. Genesis pertains to a small section of creation of the earth, not the whole earth.

Sixth, intermittent day theory. Between each 24 hr day, there were long unspecified periods of pauses. 

Seventh. Day age theory corresponds to the day age interpretation discussed above.

Eighth, analogical Day theory. Days are not 24 hr literal, but correspond to God's cycles of work and rest. There is some overlap with the proclamation day interpretation.

Ninth, framework view. Genesis 1 is a literary framework which is something of a poem rather than something of literal historical value. This corresponds to the creation poem interpretation. 

Tenth, religion only interpretation. Genesis shows who God is and is not to be taken literally. The bible is about who and why and the science is about how. This corresponds to the creation poem and Ancient Near Eastern Cosmology interpretation view.

\section{Summary}

While there are several variations and views of interpreting Genesis, the key question is whether scripture should be taken literally or figuratively. I believe in a mostly literal interpretation except when explicitly stated or implied that it is a poem or parable. Eg psalms, song of solomon, the song of Moses, and the parables of Jesus. 

Thus we can classify the above theories into a spectrum of literal vs figurative. The Young Earth Interpretation being the most literal. 

\part{Prevailing Scientific Consensus for the origin of earth and life}

Secondly, we'll want to take a look at the scientific consensus for the origin of earth and of life and also the universe. We shall also examine the basis of such arguments, papers used for such and weigh them based on the evidence available in scientific literature.

This study is of course performed with a bias towards creationism.

\section{Origin of the Universe}

\section{Origin of Planet Earth and the Solar System}

According to the article Origin of Life on Planet Earth written by Leslie E. Orgel, the prevailing thought is that earth is 4.6 billion years old. 

\subsection{Experimental Methods for Answering Questions Related to Origin of the Earth}

Ringwood 1977 also considers models for how the elemental composition in the core came to be. He also mentions how the hydrodynamic sound velocity is used to correlate or indirectly measure the composition of the core. This is rather specific but it sheds light into methods used to determine the earth's chemical composition. And knowing such experimental data, one can then confirm or reject models of how the earth's formation came to be. 

The next experimental method used in the study of geochemistry is small scale studies of oxides and metals at specific pressure and temperature (P,T) conditions. This means that elements and compounds thought to exist in the core are present on the surface, and then subject to pressures and temperatures thought to be found in the core. Nevertheless, this begs the question of how one knows the pressures and temperatures found within the core and mantle regions of the earth.

Another tool to study this question is found in seismology and vulcanology, since they reveal rarer instances of material from the mantle coming out into the earth's crust. Or in the case of seismology, elastic potential energy from the mantle is being imparted to the earth's crust and suddenly released as kinetic energy. 

\section{Origin of Life}

\subsection{literature}

One good reference here is the Origin of Life on Planet Earth Written by Leslie E. Orgel on October 1994.


\end{document}